\chapter{Improvements\label{improvements}}

In this chapter I~will discuss in detail the changes made to RIR in an attempt to bring it up to speed with GNU R byte-compiled code.


...

\section{Refactoring RIR interpreter}

As it turned out, a lot of speedup could be gained by changing the RIR bytecode interpreter. 


\todo[to compiler, to ir, to interpreter, use code snippets, describe microbenchmarks, theory (threaded code...)]
\todo[everywhere: motivation - how it helped in microbenchmarks, then how in real]
\blind

\begin{listing}[htbp]
  \caption{\label{xxx}\todo[write listing caption] microbenchmark (run with jit enable 2)}
  \begin{minted}[bgcolor=codebg]{r}
f <- function() {
    i <- 10000000L
    while (i > 0) {
        i <- i - 1
    }
}
system.time(f())[[3]]  # jit everything
t <- c()
for (x in 1:15) t <- c(t, system.time(f())[[3]])
mean(t[5:15])  # only include measurments after warmup
  \end{minted}
\end{listing}

\todo[relational operators, fast paths for logical args, unary plus minus not, loop contexts, bc cleanup, colon, superassing, inlining of instructions in main loop, threaded code, inline stack funcs, loops refactor, disable guardfuns]

\blind[15]
