The goal of this thesis was to bring RIR, an alternative bytecode compiler and interpreter for the R language, closer to R's reference impelemtation in terms of performance.

First, the R language and RIR were explored and their internal workings examined and compared. Experiments were carried out to discover their similarities and differences. Improvements to RIR infrastructure were implemented in three areas:

New bytecode instructions were added to its instruction set and employed in its compiler.

The compiler was extended to be more aware of its context and this was used to eliminate generating loop contexts in the common case when all loop control is local.

Finally, the interpreter loop was refactored and its dispatching mechanism changed from switch-based to threaded code.

All the implemented changes were evaluated using the Shootout benchmarks. Overall, the performance deficiency of RIR relative to GNU R was decreased by about 50~\%.

RIR is still under active development and further work is needed.

The benchmarks where RIR lags behind would deserve a still more thorough investigation to uncover the remaining bottlenecks.

The call mechanism of RIR is generally slower than GNU R and ways to improve it should be explored. A~way to avoid the unnecessary promise allocations for constant arguments while preserving the ideas of RIR should be found.

Still more extensions to the instruction set are possible. It could be worth looking into superinstructions (i.e. identifying what common pairs, maybe even triples of instructions occur often and combining them).

The abstract interpretation framework of RIR is another direction for future efforts. It can be improved and extended on one hand, and on the other be used to implement new analysis and optimization passes for RIR bytecode. For instance, promises could be avoided if an interprocedural analysis showed that the callee for sure evaluates its arguments.

In the early stages of this thesis, unsuccessful experiments were carried out with a tool called STOKE.\footnote{See \url{https://github.com/StanfordPL/stoke}}

STOKE is a stochastic superoptimizer for the x86-64 architecture developed at Stanford University. It tries to optimize code sequences by randomly searching the space of possible program transformations. Repeatedly applying minor changes can surprisingly produce code that runs faster than the original. New and non-obvious ways to compute equivalent results can sometimes be discovered.

STOKE is able to optimize code for performance or size, syntesize implementations from scratch and verify equivalence between different code sequences for all possible inputs.

Unfortunately, it appears that it has trouble with running on dynamically linked code, and thus could not be used directly. New ways of applying it to RIR could be explored.
