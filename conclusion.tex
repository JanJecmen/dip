The goal of this thesis was to bring RIR, an alternative bytecode compiler and interpreter for the R language, closer to R's reference impelemtation in terms of performance.

First, the R language and RIR were explored and their internal workings examined and compared. Experiments were carried out to discover their similarities and differences. Improvements to RIR infrastructure were implemented in three areas:

New bytecode instructions were added to its instruction set and employed in its compiler.

The compiler was extended to be more aware of its context and this was used to eliminate generating loop contexts in the common case when all loop control is local.

Finally, the interpreter loop was refactored and its dispatching mechanism changed from switch-based to threaded code.

All the implemented changes were evaluated using the Shootout benchmarks. Overall, the performance deficiency of RIR relative to GNU R was decreased by about 50~\%.

RIR is still under active development and further work is needed. 

% tady je tolik prace, kterou tu muzes zminit, napriklad: vic se zabirat tim, co benchmarky delaji, lepsi instrukce, vyresit ty promises v callech, zlepsit cally obecne, eager promise evaluation, etc. etc. etc. etc. 

% nesetril bych mistem a napsal tady cokoli co te napadne. 

The abstract interpretation framework of RIR is one direction for future efforts. It can be improved and extended on one hand, and on the other be used to implement new analysis and optimization passes for RIR bytecode.

In the early stages of this thesis, unsuccessful experiments were carried out with a tool for superoptimizations called Stoke.\footnote{See \url{https://github.com/StanfordPL/stoke}} New ways of applying it to RIR could be explored.

% o tomhle bych klidne napsat vic - tak 2-4 odstavce delsi to uplne v pohode snese
