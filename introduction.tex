R is a programming language that, despite being very old, seems to be steadily gaining popularity in the last years. In 2012, it was estimated \autocite{design}, that there were about 2000 package developers maintaining over 4000 packages, and over 2 million end users.

Since then, various programming language popularity rating sites report (even though they ought to be taken with a grain of salt) that R only rises.\footnote{See, e.g., \url{https://www.tiobe.com/tiobe-index/} and \url{http://pypl.github.io/PYPL.html}}

R is a very dynamic language that is easy to pick up quickly. Unfortunately, it can also be orders of magnitude slower than optimized C code and is notoriously memory hungry.

RIR is a research project at Northeastern University supervised by prof. Jan Vitek. Its long term goal is to provide a fast implementation of R through the means of its own bytecode representation, compiler and interpreter. It is designed in a way that allows for implementing analysis and optimization passes over the RIR bytecode easily.

However, at present, it is lacking the performance of the official GNU R bytecode virtual machine. To make it a viable alternative, improvements in this direction are needed.

This thesis explores where the speed difference comes from and proposes changes to be made to lower it. These changes are imlemented and evaluated.

The chapter \emph{\nameref{gnur}} gives an introduction to GNU R and discusses its features. It also goes under the hood and describes the inner workings of its interpreter and bytecode compiler.

The chapter \emph{\nameref{rir}} introduces an alternative bytecode compiler for the R language, talks about the motivation behind it, its architecture and design choices, the differences to the original and its shortcomings.

The chapter \emph{\nameref{improvements}} describes in depth the changes that were made to RIR.

The chapter \emph{\nameref{evaluation}} discusses how the performance of RIR changed after the improvements were made. It describes how the measurements were done and how the performance compares to GNU R.

The results are discussed in \emph{\nameref{conclusion}}, as well as the direction of future efforts regarding RIR.
