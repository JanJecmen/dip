\todo[write intro; cite:\newline
https://www.tiobe.com/tiobe-index/\newline
http://pypl.github.io/PYPL.html]\blind[2]

This thesis is structured in the following way:

The chapter \emph{\nameref{gnur}} gives a short introduction to GNU R and discusses its features. It also goes under the hood and describes the inner workings of its interpreter and bytecode compiler.

The chapter \emph{\nameref{rir}} introduces an alternative bytecode compiler for the R language, talks about the motivation behind it, its architecture and design choices, the differences to the original and its shortcomings.

The chapter \emph{\nameref{improvements}} describes in depth the changes that were done to RIR in this thesis.\todo[?]

The chapter \emph{\nameref{evaluation}} discusses how the performance of RIR changed after the changes done in this thesis. It describes how the measurements were done and how the performance compares to GNU R.\todo[?]

The results of this thesis are discussed in \emph{\nameref{conclusion}}, as well as the direction of future efforts regarding RIR.
