\chapter{About RIR\label{rir}}

\todo[Introduce RIR]
RIR is an alternative compiler for the R language.\footnote{Homepage: \url{https://github.com/reactorlabs/rir}} It comes with its own internal representation, an interpreter for its bytecode and an abstract interpretation framework which provides a way to easily implement static analyses on top of the RIR bytecode.

\todo[history: research project, northeastern? first appearence?]

RIR acts as a drop-in replacement for the GNU R bytecode compiler. It requires a patched version of GNU R that makes some slight adjustments that allow the standard GNU R expression evaluator function to interface with the RIR bytecode compiler and interpreter. RIR is written in C and C++ and is compiled as a shared library that can be dynamically loaded by R.

The architecture is very similar to GNU R. The compiler is 

\todo[write about rir bytecode]

\todo[how is rir bc different]

\todo[optimizations, ai framework...]

\todo[guards]

\todo[cpool and srcpool]

% % % % % % % % % % % % % % % % % % % % % % % % % % % % % % 

\section{Why is RIR slow}

\todo[..., calls to ast interpreter, profiling, optimizations, allocations]
