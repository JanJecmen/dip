% arara: xelatex: { shell: yes }
% arara: makeglossaries
% arara: biber
% arara: xelatex: { shell: yes }
% arara: xelatex: { shell: yes }

\documentclass[thesis=M,english,hidelinks]{template/FITthesisXE}

\usepackage{graphicx}	% graphics files inclusion
\usepackage{dirtree}	% directory tree visualisation
\usepackage{longtable}	% tables which Pandoc use
\usepackage{lscape}		% to be able to rotate stuff
\usepackage{metalogo}	% for \XeLaTeX
\usepackage{xcolor}
\usepackage{blindtext}

\newcommand{\todo}[1]{\textcolor{red}{\textbf{[[#1]]}}}
\newcommand{\blind}[1][1]{\textcolor{gray}{\Blindtext[#1][1]}}

\bibliography{library.bib}

\makeglossaries
\newacronym{API}{API}{Application programming interface}
\newacronym{CLI}{CLI}{Command Line Interface}
\newacronym{CRAN}{CRAN}{The Comprehensive R Archive Network}
\newacronym{GNU}{GNU}{GNU's Not Unix!}

\glsaddall	% add even unused acronyms

% % % % % % % % % % % % % % % % % % % % % % % % % % % % % % 

\acknowledgements{\todo[acknowledgements]
\blind[1]
}
\abstractEN{R is a dynamic programming language that, despite being over 20 years old, is still widely used. RIR is an alternative to its bytecode compiler and interpreter that aims to facilitate adding static analyses and optimization passes easily. RIR is under development and does not currently match the performance of standard R. This thesis attempts to amend the situation. It extends the RIR internal representation, adds new features to its compiler and refactors its interpreter. The average slowdown versus standard R is brought down by about one half in the Shootout benchmarks.
}
\abstractCS{R je dynamicý programovací jazyk, navzdory svému stáří dnes stále oblíbený. RIR je alternativní implementace kompilátoru a interpretu R bajtkódu, která umožňuje snadno provádět statickou analýzu a přidávat optimalizace. RIR je ve vývoji a zatím nedosahuje výkonu standardního R. Tato diplomová práce se pokouší přiblížit výkon RIR k~výkonu standardního R. V~jejím rámci byly přidány nové instrukce do RIR bajtkódu a nová funkcionalita do jeho kompilátoru a došlo k~přepracování jeho interpretu. Průměrné zpomalení na sadě benchmarků Shootout proti standardnímu R bylo sníženo o~polovinu.
}
\title{Improvements of the RIR bytecode toolchain}
\authorGN{Jan}
\authorFN{Ječmen}
\authorWithDegrees{Bc. Jan Ječmen}
\author{Jan Ječmen}
\supervisor{Ing. Petr Máj}
\keywordsEN{R language, RIR, bytecode compiler \& interpreter, optimizations}
\keywordsCS{jazyk R, RIR, kompilátor \& interpret bajtkódu, optimalizace}
\department{Department of Theoretical Computer\newline Science}
\placeForDeclarationOfAuthenticity{Prague}
\declarationOfAuthenticityOption{1}
\website{https://github.com/JanJecmen/dip}
\assignment{assignment.pdf}


\begin{document}

\hyphenation{frame-work}


\begin{introduction}
\label{introduction}
\todo[write intro]

\todo[cite: https://www.tiobe.com/tiobe-index/
http://pypl.github.io/PYPL.html]

\blind[3]
\todo[write about structure]
\blind[1]

\end{introduction}

\chapter{Foo \label{foo}}

\todo{write chapter}
\blind[2]

\section{Section bar}\label{bar}

\todo{write section}
\blind[2]

\subsection{Subsection baz}\label{baz}

\todo{write subsection}
\blind[2]

\begin{listing}[htbp]
\caption{\label{xxx}code}
\begin{minted}[bgcolor=codebg]{r}
f <- rir.compile(function(x) {
    i <- 0
    while (i < x) {
        cat(i)
    }
})
f(10)
\end{minted}
\end{listing}

\url{https://github.com/reactorlabs/rir}

Here is citation demo: Flask \autocite{flask}.


\begin{conclusion}
\label{conclusion}
\todo[conclusion, future work, related work, fails - stoke etc.]

\end{conclusion}

\printbibliography

\appendix

\chapter{Acronyms}
\printglossary[type=\acronymtype,style=acronyms]

\chapter{Contents of the enclosed CD}

\vfill

\todo[contents of cd]
\blind[1]

%\begin{dirfigure}%
%    \dirtree{%
%        .1 README.md\DTcomment{stručný popis obsahu média}.
%        .1 src.
%            .2 utvsapi-benchmark\DTcomment{skripty pro měření rychlosti}.
%                .3 logs\DTcomment{záznamy z~měření rychlosti}.
%            .2 utvsapi-django\DTcomment{implementace ukázkové služby v~DRF}.
%            .2 utvsapi-eve\DTcomment{implementace ukázkové služby v~Eve}.
%            .2 utvsapi-ripozo\DTcomment{implementace ukázkové služby v~ripozu}.
%            .2 utvsapi-sandman\DTcomment{implementace ukázkové služby v~sandmanu2}.
%            .2 utvsapitoken\DTcomment{společný modul pro práci s~tokenem}.
%            .2 diplomka\DTcomment{zdrojová forma práce ve formátu Markdown a \XeLaTeX{}}.
%        .1 DP\_Hroncok\_Miroslav\_2016.pdf\DTcomment{text práce ve formátu PDF}.
%    }
%\caption{Obsah přiloženého média}
%\end{dirfigure}


\end{document}
