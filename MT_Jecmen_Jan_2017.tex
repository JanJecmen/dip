% arara: xelatex: { shell: yes }
% arara: makeglossaries
% arara: biber
% arara: xelatex: { shell: yes }
% arara: xelatex: { shell: yes }

\documentclass[thesis=M,english,hidelinks]{template/FITthesisXE}

\usepackage{graphicx}	% graphics files inclusion
\usepackage{dirtree}	% directory tree visualisation
\usepackage{longtable}	% tables which Pandoc use
\usepackage{lscape}		% to be able to rotate stuff
\usepackage{metalogo}	% for \XeLaTeX
\usepackage{xcolor}
\usepackage{blindtext}

\newcommand{\todo}[1]{\textcolor{red}{\textbf{[[#1]]}}}
\newcommand{\blind}[1][1]{\textcolor{gray}{\Blindtext[#1][1]}}

\bibliography{library.bib}

\makeglossaries
\newacronym{API}{API}{Application programming interface}
\newacronym{ASCII}{ASCII}{American Standard Code for Information Interchange}
\newacronym{CLI}{CLI}{Command Line Interface}
\newacronym{GNU}{GNU}{GNU's Not Unix!}
\newacronym{ISO}{ISO}{International Organization for Standardization}

\glsaddall	% add even unused acronyms

% % % % % % % % % % % % % % % % % % % % % % % % % % % % % % 

\acknowledgements{\todo[acknowledgements]
\blind[1]
}
\abstractEN{R is a dynamic programming language that, despite being over 20 years old, is still widely used. RIR is an alternative to its bytecode compiler and interpreter that aims to facilitate adding static analyses and optimization passes easily. RIR is under development and does not currently match the performance of standard R. This thesis attempts to amend the situation. It extends the RIR internal representation, adds new features to its compiler and refactors its interpreter. The average slowdown versus standard R is brought down by about one half in the Shootout benchmarks.
}
\abstractCS{R je dynamicý programovací jazyk, navzdory svému stáří dnes stále oblíbený. RIR je alternativní implementace kompilátoru a interpretu R bajtkódu, která umožňuje snadno provádět statickou analýzu a přidávat optimalizace. RIR je ve vývoji a zatím nedosahuje výkonu standardního R. Tato diplomová práce se pokouší přiblížit výkon RIR k~výkonu standardního R. V~jejím rámci byly přidány nové instrukce do RIR bajtkódu a nová funkcionalita do jeho kompilátoru a došlo k~přepracování jeho interpretu. Průměrné zpomalení na sadě benchmarků Shootout proti standardnímu R bylo sníženo o~polovinu.
}
\title{Improvements of the RIR bytecode toolchain}
\authorGN{Jan}
\authorFN{Ječmen}
\authorWithDegrees{Bc. Jan Ječmen}
\author{Jan Ječmen}
\supervisor{Ing. Petr Máj}
\keywordsEN{R language, RIR, bytecode compiler \& interpreter, optimizations}
\keywordsCS{jazyk R, RIR, kompilátor \& interpret bajtkódu, optimalizace}
\department{Department of Theoretical Computer\newline Science}
\placeForDeclarationOfAuthenticity{Prague}
\declarationOfAuthenticityOption{1}
\website{https://github.com/JanJecmen/dip}
\assignment{assignment.pdf}


\begin{document}

\input{hyphenation.tex}

\begin{introduction}
\label{introduction}
R is a programming language that, despite being very old, seems to be steadily gaining popularity in the last years. In 2012, it was estimated \autocite{design}, that there were about 2000 package developers maintaining over 4000 packages, and over 2 million end users.

Since then, various programming language popularity rating sites report (even though they ought to be taken with a grain of salt) that R only rises.\footnote{See, e.g., \url{https://www.tiobe.com/tiobe-index/} and \url{http://pypl.github.io/PYPL.html}}

R is a very dynamic language that is easy to pick up quickly. Unfortunately, it can also be orders of magnitude slower than optimized C code and is notoriously memory hungry.

RIR is a research project at Northeastern University supervised by prof. Jan Vitek. Its long term goal is to provide a fast implementation of R through the means of its own bytecode representation, compiler and interpreter. It is designed in a way that allows for implementing analysis and optimization passes over the RIR bytecode easily.

However, at present, it is lacking the performance of the official GNU R bytecode virtual machine. To make it a viable alternative, improvements in this direction are needed.

This thesis explores where the speed difference comes from and proposes changes to be made to lower it. These changes are implemented and evaluated.

The chapter \emph{\nameref{gnur}} gives an introduction to the R language and discusses its features. It also goes under the hood and describes the inner workings of its interpreter and bytecode compiler -- the GNU R.

The chapter \emph{\nameref{rir}} introduces an alternative bytecode compiler for the R language, talks about the motivation behind it, its architecture and design choices, the differences to the original and its shortcomings.

The chapter \emph{\nameref{improvements}} describes in depth the changes that were made to RIR.

The chapter \emph{\nameref{evaluation}} discusses how the performance of RIR changed after the improvements were made. It describes how the measurements were done and how the performance compares to GNU R.

The results are discussed in \emph{\nameref{conclusion}}, as well as the direction of future efforts regarding RIR.

\end{introduction}

\chapter{Foo \label{foo}}

\todo[write chapter]
\blind[2]

\section{Section bar}\label{bar}

\todo[write section]
\blind[2]

\subsection{Subsection baz}\label{baz}

\todo[write subsection]
\blind[2]
\todoimg{0.8}

\blind[2]

\begin{listing}[htbp]
  \begin{minted}[bgcolor=codebg]{r}
f <- rir.compile(function(x) {
    i <- 0
    while (i < x) {
        cat(i)
    }
})
f(10)
  \end{minted}
  \caption{\label{xxx}\todo[write listing caption]}
\end{listing}

\blind[2]

\url{https://github.com/reactorlabs/rir}

Here is citation demo: Flask \autocite{flask}.

\blind[2]


\begin{conclusion}
\label{conclusion}
\todo[conclusion, future work, related work, fails - promises of consts, stoke etc.]

\end{conclusion}

\printbibliography

\appendix

\chapter{Acronyms}
\printglossary[type=\acronymtype,style=acronyms]

\chapter{Contents of the enclosed CD}

\vfill

\begin{dirfigure}%
   \dirtree{%
       .1 README.txt.
       .1 thesis\DTcomment{sources for the thesis in \XeLaTeX{}}.
           .2 benchmarks\DTcomment{scripts to measure and plot the benchmarks, measured data}.
           .2 images\DTcomment{plots used in the thesis}.
       .1 MT\_Jecmen\_Jan\_2017.pdf\DTcomment{PDF of the thesis}.
       .1 revisions.txt\DTcomment{list of revisions that added new features}.
   }
\caption{Contents of the enclosed CD}
\end{dirfigure}


\end{document}
